\begin{sidewaystable}
\footnotesize
\caption{Statistics for resolved system ``RAS DATA SET 2'', costing \$9235.}
\centering
\begin{tabular}{c||c|c||c|c|c|c|c||c|c|c}
  \hline \hline
  &
  Unpref. & 
  Delay &
  Node &
  When &
  SA &
  +/- &
  Pty &
  TWT &
  +/- &
  Pty \\
      \hline
      \multirow{2}{*}{A1} &
      \multirow{2}{*}{4} &
      \multirow{2}{*}{2760} &
      37 &
      5842.5 &
      3600 &
        -2242.5 &
        0 &
      \multirow{2}{*}{5400} &
        \multirow{2}{*}{-3880.5} &
        \multirow{2}{*}{0}
      \\
      \cline{4-8}
       &
       &
       &
      39 &
      9280.5 &
      7800 &
        -1480.5 &
        0 &
      
         &
        
      \\
      \hline
      \multirow{2}{*}{A2} &
      \multirow{2}{*}{0} &
      \multirow{2}{*}{240} &
      37 &
      10636.424 &
      11400 &
        763.576 &
        0 &
      \multirow{2}{*}{12600} &
        \multirow{2}{*}{-713.692} &
        \multirow{2}{*}{0}
      \\
      \cline{4-8}
       &
       &
       &
      39 &
      13313.692 &
      15600 &
        2286.308 &
        0 &
      
         &
        
      \\
      \hline
      \multirow{2}{*}{A3} &
      \multirow{2}{*}{4} &
      \multirow{2}{*}{0} &
      37 &
      17095.5 &
      18000 &
        904.5 &
        0 &
      \multirow{2}{*}{19800} &
        \multirow{2}{*}{-346.5} &
        \multirow{2}{*}{0}
      \\
      \cline{4-8}
       &
       &
       &
      39 &
      20146.5 &
      22200 &
        2053.5 &
        0 &
      
         &
        
      \\
      \hline
      \multirow{2}{*}{A4} &
      \multirow{2}{*}{22} &
      \multirow{2}{*}{60} &
      37 &
      36347.461 &
      31800 &
        -4547.461 &
        0 &
      \multirow{2}{*}{39000} &
        \multirow{2}{*}{-959.375} &
        \multirow{2}{*}{0}
      \\
      \cline{4-8}
       &
       &
       &
      0 &
      39959.375 &
      36600 &
        -3359.375 &
        0 &
      
         &
        
      \\
      \hline
      \multirow{2}{*}{B1} &
      \multirow{2}{*}{9} &
      \multirow{2}{*}{7260} &
      37 &
      14048.112 &
      4800 &
        -9248.112 &
        113 &
      \multirow{2}{*}{9600} &
        \multirow{2}{*}{-7798.224} &
        \multirow{2}{*}{0}
      \\
      \cline{4-8}
       &
       &
       &
      39 &
      17398.224 &
      9000 &
        -8398.224 &
        66 &
      
         &
        
      \\
      \hline
      \multirow{2}{*}{B2} &
      \multirow{2}{*}{16} &
      \multirow{2}{*}{120} &
      37 &
      32423.625 &
      26400 &
        -6023.625 &
        0 &
      \multirow{2}{*}{35400} &
        \multirow{2}{*}{-777.375} &
        \multirow{2}{*}{0}
      \\
      \cline{4-8}
       &
       &
       &
      39 &
      36177.375 &
      31200 &
        -4977.375 &
        0 &
      
         &
        
      \\
      \hline
      \multirow{2}{*}{B3} &
      \multirow{2}{*}{8} &
      \multirow{2}{*}{0} &
      37 &
      7313.144 &
      11400 &
        4086.856 &
        0 &
      \multirow{2}{*}{10800} &
        \multirow{2}{*}{60.425} &
        \multirow{2}{*}{0}
      \\
      \cline{4-8}
       &
       &
       &
      0 &
      10739.575 &
      16800 &
        6060.425 &
        0 &
      
         &
        
      \\
      \hline
      \multirow{2}{*}{C1} &
      \multirow{2}{*}{0} &
      \multirow{2}{*}{60} &
      37 &
      35829.006 &
      26400 &
        -9429.006 &
        123 &
      \multirow{2}{*}{39000} &
        \multirow{2}{*}{-134.089} &
        \multirow{2}{*}{0}
      \\
      \cline{4-8}
       &
       &
       &
      39 &
      39134.089 &
      31800 &
        -7334.089 &
        7 &
      
         &
        
      \\
      \hline
      \multirow{2}{*}{C2} &
      \multirow{2}{*}{0} &
      \multirow{2}{*}{0} &
      37 &
      198 &
      0 &
        -198 &
        0 &
      \multirow{2}{*}{3600} &
        \multirow{2}{*}{30} &
        \multirow{2}{*}{0}
      \\
      \cline{4-8}
       &
       &
       &
      39 &
      3570 &
      5400 &
        1830 &
        0 &
      
         &
        
      \\
      \hline
      \multirow{2}{*}{C3} &
      \multirow{2}{*}{0} &
      \multirow{2}{*}{22433.141} &
      37 &
      59181.43 &
      20400 &
        \multicolumn{2}{|c||}{N/A} &
      \multirow{2}{*}{25200} &
        \multicolumn{2}{c}{\multirow{2}{*}{N/A}}
      \\
      \cline{4-8}
       &
       &
       &
      0 &
      63784.718 &
      25800 &
        \multicolumn{2}{|c||}{N/A} &
      
        
      \\
      \hline
      \multirow{2}{*}{D1} &
      \multirow{2}{*}{0} &
      \multirow{2}{*}{0} &
      37 &
      40216.5 &
      43800 &
        3583.5 &
        0 &
      \multirow{2}{*}{44400} &
        \multicolumn{2}{c}{\multirow{2}{*}{N/A}}
      \\
      \cline{4-8}
       &
       &
       &
      39 &
      44722.5 &
      51000 &
        \multicolumn{2}{|c||}{N/A} &
      
        
      \\
      \hline
      \multirow{2}{*}{D2} &
      \multirow{2}{*}{27} &
      \multirow{2}{*}{1800} &
      37 &
      2179.383 &
      3600 &
        1420.617 &
        0 &
      \multirow{2}{*}{6600} &
        \multirow{2}{*}{-1361.531} &
        \multirow{2}{*}{0}
      \\
      \cline{4-8}
       &
       &
       &
      39 &
      7961.531 &
      9600 &
        1638.469 &
        0 &
      
         &
        
      \\
      \hline
      \multirow{2}{*}{E1} &
      \multirow{2}{*}{0} &
      \multirow{2}{*}{43200} &
      37 &
      48820.912 &
      6600 &
        \multicolumn{2}{|c||}{N/A} &
      \multirow{2}{*}{9600} &
        \multicolumn{2}{c}{\multirow{2}{*}{N/A}}
      \\
      \cline{4-8}
       &
       &
       &
      39 &
      54087.822 &
      14400 &
        \multicolumn{2}{|c||}{N/A} &
      
        
      \\
      \hline
      \multirow{2}{*}{E2} &
      \multirow{2}{*}{0} &
      \multirow{2}{*}{0} &
      37 &
      1134.858 &
      1800 &
        665.142 &
        0 &
      \multirow{2}{*}{7200} &
        \multirow{2}{*}{-26.294} &
        \multirow{2}{*}{0}
      \\
      \cline{4-8}
       &
       &
       &
      0 &
      7226.294 &
      10800 &
        3573.706 &
        0 &
      
         &
        
      \\
      \hline
      \multirow{2}{*}{E3} &
      \multirow{2}{*}{0} &
      \multirow{2}{*}{43200} &
      37 &
      97546.285 &
      9000 &
        \multicolumn{2}{|c||}{N/A} &
      \multirow{2}{*}{12000} &
        \multicolumn{2}{c}{\multirow{2}{*}{N/A}}
      \\
      \cline{4-8}
       &
       &
       &
      0 &
      105575.14 &
      18000 &
        \multicolumn{2}{|c||}{N/A} &
      
        
      \\
      \hline
      \multirow{2}{*}{E4} &
      \multirow{2}{*}{32} &
      \multirow{2}{*}{0} &
      37 &
      27424.521 &
      30000 &
        2575.479 &
        0 &
      \multirow{2}{*}{32400} &
        \multirow{2}{*}{-269.3} &
        \multirow{2}{*}{0}
      \\
      \cline{4-8}
       &
       &
       &
      0 &
      32669.3 &
      37800 &
        5130.7 &
        0 &
      
         &
        
      \\
      \hline
      \multirow{2}{*}{F1} &
      \multirow{2}{*}{39} &
      \multirow{2}{*}{840} &
      37 &
      25268.18 &
      29400 &
        4131.82 &
        0 &
      \multirow{2}{*}{33600} &
        \multirow{2}{*}{1.638} &
        \multirow{2}{*}{0}
      \\
      \cline{4-8}
       &
       &
       &
      39 &
      33598.362 &
      41400 &
        7801.638 &
        0 &
      
         &
        
      \\
      \hline
      \multirow{2}{*}{F2} &
      \multirow{2}{*}{0} &
      \multirow{2}{*}{34560} &
      37 &
      63080.573 &
      30000 &
        \multicolumn{2}{|c||}{N/A} &
      \multirow{2}{*}{36000} &
        \multicolumn{2}{c}{\multirow{2}{*}{N/A}}
      \\
      \cline{4-8}
       &
       &
       &
      0 &
      76786.29 &
      51000 &
        \multicolumn{2}{|c||}{N/A} &
      
        
      \\
\end{tabular}
\label{table:RDS2-9235.tex} 
\end{sidewaystable}