\documentclass[10pt,a4paper,final]{article}
\usepackage[utf8x]{inputenc}
\usepackage{ucs}
\usepackage[english]{babel}
\usepackage{multirow}
\usepackage{rotating}
\author{Lukáš Petrovický}
\title{RAS2012: Competition Entry}
\begin{document}

\maketitle

\begin{abstract}
This article describes an entry to the \emph{2012 RAS Problem Solving Competition}, concerning dispatching on multi-track territories. The entry is based on Drools Planner, a Java-based solver. On a reasonably recent computer, the resulting algorithm is able to produce feasible solutions within seconds and has been fine-tuned to provide best results in under 3 minutes. Source code of the entry is open and well documented.
\end{abstract}

\section{Introduction}

Working in the field of expert systems, tackling complicated problems can easily become one's hobby. But even in one's spare time, one still wants to find out how one compares to others. Having heard of the \emph{2012 RAS Problem Solving Competition} at that point, registering wasn't at all a tough decision. After spending many a night on it, this is what emerged.

We first describe the core concepts behind our submission, following by notes on the actual technical implementation. We show how the application can be used for both its primary purpose and also additional ones. We discuss the achieved results and in the appendices, we provide statistics for the best resolved systems.

Not being members of any of the Operations Research communities or even academia at all, we sincerely apologize to the reader of this paper for unknowingly not following any and all conventions for writing such documents. The seasoned researcher must be frightened by the complete lack of citations and sourcing~-- however, there is a simple explanation: we used no literature whatsoever in our process, except for the occasional software documentation on-line.

Throughout this text, we will use the following words, meaning:

\begin{description}
\item[Algorithm] is the solver-related code and configuration being submitted.
\item[Application] consists of the algorithm plus other non-essential code, such as the visualizers.
\item[Solution] is a resolved system. It can either be feasible, having all the hard constraints met, or unfeasible.
\item[Submission] consists of this paper, the example solutions and the application. It is our final output for the purposes of the competition.
\end{description}

\section{Key concepts}

In this section, we describe the various approaches we have taken to implement the rules and constraints required by the competition. These are the key concepts leveraged in the algorithm, whose implementation will be described in later sections.

\begin{description}

\item[Meet-pass events] The most important concept to grasp in our case is the meet-pass event, or more precisely the lack thereof. The route of every train is planned independently of every other train and the only relevant criteria is the overall system's efficiency expressed by its cost.\footnote{While, of course, still honoring every rule of the competition.} This inevitably leads to trains being sent onto a collision course. These situations are eliminated by enforcing the five-minute separation rule at each arc's entry points.

\item[Train routes] From each particular data set, a territory is built. This territory is then analyzed and every possible route through that territory is found. Each train holds a set of routes it is allowed to take, from which solver will attempt to pick the best one\footnote{The competition rules state that the algorithm cannot assign trains to sidings. A case could be made that our approach here violates this rule~-- the algorithm indeed may put a train on a route with a siding. However, as the solution improves, this situation will be reduced to only occurring during what would be seen as a meet-pass event. Strictly speaking, we don't violate the rule at all, since we don't assign trains to sidings~-- we assign routes to trains.}. A train is not allowed to take a particular route when any of the following conditions hold true:

\begin{itemize}
\item Train is required to pass through a node that is not on the route. Examples would include situations when the selected route doesn't include the node where the train has a schedule adherence requirement set.
\item Length of the train is bigger than the length of any one siding on the route.
\item There are sidings on the route of a train carrying hazardous materials. This is to fulfill the requirement that such train can never take a siding.
\item There are sidings on the route of a heavy train. This is the easiest way to fulfill the requirement that a heavy train must never use a siding when meet-passing another NSA train.
\end{itemize}

\item[Train wait points] Our solution only allows trains to stop at specific points on the route, called ``wait points''\footnote{If we allowed the train to stop at any node on the route, we would be significantly extending the state space. Benchmarking the algorithm showed that it wouldn't be beneficial to the quality of the solution. We have therefore made a decision to only stop the trains where they don't block main tracks.}. A train can be held when its lead engine enters a node for which any of the following conditions holds true:

\begin{itemize}
\item The node is the source terminal of the train. The train is not in the territory yet.
\item The node is the starting node of a crossover arc, keeping the other main track available for whatever trains to pass.
\item The node is the ending node of a siding arc. The train occupies the siding.
\end{itemize}

Trains are automatically stopped for maintenance windows. However, no additional delays are allowed in those nodes~-- the train starts to move immediately after the maintenance window is over.

\item[Train wait times] In each of the wait points described above, the algorithm may incur one of a selection of delays. These delays are called ``wait times'' and range from 1 minute to $X$ minutes, where $X$ is the minimal possible delay that would cause the train to fall outside the planning horizon waiting\footnote{This is done to once again reduce the state space by eliminating wait times that aren't viable. Once a train falls out of the planning horizon, we no longer care about its further journey.}. The wait times aren't distributed evenly across the stated range~-- they get sparser as we progress towards bigger numbers in the range. That allows for both fine-grained and coarse-grained planning, while still keeping the number of possibilities relatively small.

\item[Itinerary] An itinerary of a train consists of a route and 0 or more wait times at the wait points on that route.

\item[Solution] A solution is a set of itineraries, one for each train. The best found solution becomes a resolved system. A solution differs from the other by either of the trains taking a different route, waiting at different points and/or for different times.

\end{description}

\section{Implementation}

In this section, we will describe the implementation of the algorithm, its various properties, strengths and drawbacks.

\subsection{Drools Planner} 

Drools Planner\footnote{The Drools project, http://www.jboss.org/drools/} is an open-source Java-based solver that currently implements various tabu search techniques and simulated annealing. It allows for stating your problem in a declarative manner, using the following basic concepts:

\begin{description}
\item[Planning entity] in our case is the itinerary for a particular train.

\item[Planning variable] is a part of the planning entity. By changing variable's value, Planner can arrive at a better or worse solution. In our case, we have two planning variables: the route for the train and wait times in wait points.

\item[Planning values] are possible values to apply to a planning variable at any given time. For example, the itinerary has a list of routes that are available for its train to take. Each of them can become a value of the planning variable, resulting in a change in solution quality.

\item[Move] is a way to change one solution into another, to cross from one state to another in a state space. Usually a move consists of changing the value of a planning variable. In our case, we know the following moves:

\begin{itemize}
\item A move to assign or remove a wait time for a specific wait point in the itinerary.
\item A move to change a route the train will take. We keep the existing wait times, except obviously when their respective wait point isn't part of the new route.
\end{itemize}

\item[Scoring function] is a means of evaluating the quality of the solution. In our case, it consists of two parts, the hard and the soft score. The soft score represents the cost of the system as defined by the competition. The hard score is a sum of five-minute rule violations and the number of trains taking sidings where they do not wait. In a feasible solution, the hard score will be zero, meaning that each train obeys the five-minute rule and never takes a siding unless it also stops there.

Scoring function in Planner can take two forms - first, you can write it in plain old Java. Alternatively, you can leverage Drools Expert, giving at your disposal the full power of a ReteOO-based inference engine. We use the former\footnote{Implementing an efficient scoring function without the help of an inference engine means writing a lot of caching and also figuring out a mechanism to only incrementally update those parts of the score where variables actually changed. Those features you get ``for free'' from a system such as Drools Expert; however, the rules of the scoring function ended up being so complex that it actually justified writing code in the imperative Java language.}.

\end{description}

When all these are provided to Planner, along with a solver configuration where we specify  the types of algorithms to use and termination conditions\footnote{Our algorithm is configured to terminate after it ran for 3 minutes and didn't improve the solution in the last 1000 steps.}, it will start traversing the state space by trying the moves, looking for the best solution. While solving, Planner makes no I/O operations, utilizing only the CPU and RAM~-- which means that any performance improvement to the scoring function, to the moves or to Planner itself will directly affect the quality of your scores. 

One of the benefits of Planner is that it is a Java application. That means neither the developer of the solver nor its user is limited to any particular operating system or platform. For our development and testing, we've been using Linux, but we've seen the submission perform on Windows as well.

At the moment, Planner is a single-threaded application. Using multi-core CPUs will yield no benefit performance-wise, although you can leverage them to run multiple solvers at once~-- resolving one data set per core.

\subsection{Typical flow of the algorithm}

\begin{enumerate}
\item Start processing.
\item Parse the data set, re-constructing the territory.
\item Turn the data set into an initial solution. Each train is assigned the fastest possible route through the territory with no wait times. 
\item Launch the solver with the initial solution. The solver then iterates:

\begin{enumerate}
\item Generate all possible moves for the solution.
\item Based on the user-specified solver configuration, some or all of the moves are evaluated. This is where the simulated annealing and tabu searches come into play.
\item Should any of the moves result in a better solution, store it.
\item Repeat.
\end{enumerate}

\item When a user-specified termination condition is reached, the solver ends and reports the best found solution. The termination condition in our case is achieving 3 minutes of run time.
\item We take the solution and convert it into the submission format. Optionally, we visualize and tabulate.
\item End processing.
\end{enumerate}

\subsection{Additional features} 

During the development, debugging and tuning of the algorithm, we have developed various tools to make our work easier\footnote{As a side note, although technically not a ``feature,'' the algorithm is written in such a way that it should be very easy to adapt it for changing circumstances. For example, changing the territory to a directed graph by introducing one-way arcs should only take a couple minutes and shouldn't affect performance.}. Following is a list of them.

\begin{description}

\item[Data set parser] is generated on-the-fly from a JavaCC-based\footnote{JavaCC parser/scanner generator, http://javacc.java.net/} grammar and can be easily extended or reused in other applications.

\item[Basic visualizations] of train routes, territories and solutions are possible. We draw graphs of these to help with the analysis of the resolved systems and debugging of the algorithm.

\item[Solution validation] for reading the resolved system's XML, calculating its cost and also visualizing it. Can be used for comparing various competition submissions against each other using a single algorithm.

\end{description}


\subsection{Usage}

\subsubsection{Compiling}

We've relied on the industry-standard Maven\footnote{Apache Maven, http://maven.apache.org/} build tool for our dependency management and release needs. It is open-source and available for every major operating system. Our source code is tracked on Github\footnote{Our homepage, http://triceo.github.com/ras12/} using the Git SCM.

Once you have downloaded the Git and Maven tools, download the source code by executing the command (see Figure \ref{figure:git}). This will create a \texttt{ras12} folder in your current working directory, containing a one-for-one copy of our entire code repository. After switching into it, simply type another command (see Figure \ref{figure:maven}).

Maven starts downloading our dependencies, building our code and running our extensive tests. It may take a while and it should end in success. When it's done, switch to a directory called \texttt{ras2012-solver}. You will notice a new file called \texttt{target/ras2012-solver-1.0-jar-with-dependencies.jar}. You can use it to run the solver, as described in the following section.

\begin{figure}
\centering
\texttt{git clone git://git@github.com:triceo/ras12.git}
\caption{Obtaining the source code.}
\label{figure:git}
\end{figure}

\begin{figure}
\centering
\texttt{mvn install}
\caption{Building the binary distribution.}
\label{figure:maven}
\end{figure}

\begin{figure}
\centering
\texttt{java -jar <JAR-FILE> -r -d <PATH-TO-DATA-SET> -x <SEED>}
\caption{Running the application in a resolver mode, optionally specifying a random seed.}
\label{figure:run-resolver}
\end{figure}

\begin{figure}
\centering
\texttt{java -jar <JAR-FILE> -e -d <PATH-TO-DATA-SET> -s <PATH-TO-XML>}
\caption{Running the application in an evaluation mode.}
\label{figure:run-evaluation}
\end{figure}

\begin{figure}
\centering
\texttt{java -jar <JAR-FILE> -l}
\caption{Running the application in lookup mode.}
\label{figure:run-lookup}
\end{figure}

At the time of submission, our application relies on a yet unreleased version of Drools Planner, directly from the master branch. Thus, we recommend using the compiled binary version of our application provided as part of our submission, since that is guaranteed to work. In case the reader wishes to compile the code for themselves, they may first have to resolve various compile-time or runtime problems arising from possible incompatibilities between the current Planner version and the future version being used.

\subsubsection{Running}

To start resolving a data set, see Figure \ref{figure:run-resolver}. To calculate cost for a specific resolved system, see Figure \ref{figure:run-evaluation}. To verify how your solver configuration performs on the three provided data sets, see Figure \ref{figure:run-lookup}. It will run the algorithm many times with a fresh random seed each time and provide a summary of the results, much like in Table \ref{table:result} and Figure \ref{figure:plot}.

Please note that in all the figures, \texttt{<JAR-FILE>} refers to the \texttt{.jar} file from the previous section, or the one received as part of this submission.

\section{Achieved results}

Running the algorithm 50 times over each data set and using a fresh random seed every time, we have compiled a set of results, see Table \ref{table:result}. All these were reached within 3 minutes in a single-threaded run, using Intel i7 Q820 processor running Fedora 17 and 2 GB of heap space inside Java 7 runtime environment.

As we can see, the differences between the best and the worst scores produced by the algorithm are fairly large~-- even more than 100 \% for the smallest data set. This randomness can be eliminated by providing a fixed seed. That may not be practical however, since various data sets behave differently with different seeds. Best practice could be to use the median (``Q2'') value as a guide for what to expect from the algorithm generally.

Please note that via the benchmarking functionality of Drools Planner\footnote{We had to improve that functionality inside Planner to allow for parallel benchmarking. Otherwise our benchmarks would probably still be running today.}, users may be able to fine-tune the algorithm to be focused either on providing better solutions or on faster turnaround times. Drools Planner even allows for retrieving the intermediate results of the algorithm and modifying the problem while it's being solved, which makes it the ideal tool for real-time planning.

For statistics of the resolved systems, see tables \ref{table:RASDATASET1}, \ref{table:RASDATASET2} and \ref{table:RASDATASET3} respectively. 

\begin{figure}
\centering
\includegraphics[width=120mm]{chart.png}
\caption{Plotting the data for the various resolved systems.}
\label{figure:plot}
\end{figure}

\begin{table}
\footnotesize
\caption{Performance per data set.}
\centering
\begin{tabular}{c||c|c|c|c|c}
\hline \hline
                 & Best                 & Q1                 & Q2                & Q3                & Worst \\ 
\hline
RDS1 & \$1199   & \$1800   & \$1958  & \$2190  & \$2761 \\
\hline
RDS2 & \$7955   & \$9406   & \$10064  & \$11576  & \$13230 \\
\hline
RDS3 & \$11306   & \$12790   & \$13871.5  & \$14968  & \$16657 \\
\end{tabular} 
\label{table:result} 
\end{table}

\section{Conclusion}

We are submitting a solution to the stated problem that is easy to understand and extend and doesn't limit the user to any particular computing platform. We allow for tabulating and visualizing resolved systems for easier analysis, be they ours or competitors'.

On all the provided data sets, the algorithm reaches feasible solutions within seconds. We provide resolved systems that were reached within 3 minutes of starting the algorithm. However, the algorithm can be easily adapted to run continuously and provide results on-demand.

The application is configured to work reasonably well for the three real-world data sets\footnote{Every plot and table in this paper has been generated by the algorithm on-the-fly.}. Users are encouraged to benchmark their own solver configurations based on their problem sizes, in order to maximize the algorithm's performance for their particular use case.

API of the algorithm is well-documented and contains an extensive test suite. The output of both, as well as the source code, has been published at the submission's website.

\appendix

Following are appendixes where we provide statistics of the best available resolved systems. The paper would be complete even without them, for which reason we believe they shouldn't be counted toward the 10-page limit.

\section{Resolved systems}

In this section, we show the best solutions reached for each data set. 

\begin{sidewaystable}
\footnotesize
\caption{Resolved system ``RAS DATA SET 1'', costing \$1829. Seed: -842782272.}
\centering
\begin{tabular}{c||c|c|c||c|c|c|c||c|c|c}
  \hline \hline
  &
  Unpref. & 
  Delay &
  Pty &
  Node &
  SA &
  Actual &
  Pty &
  TWT &
  Actual &
  Pty \\
      \hline
      \multirow{2}{*}{A1} &
      \multirow{2}{*}{0} &
      \multirow{2}{*}{0} &
      \multirow{2}{*}{0} &
      37 &
      3600 &
        2398.5 &
        0 &
      \multirow{2}{*}{5400} &
        \multirow{2}{*}{4945.5} &
        \multirow{2}{*}{0}
      \\
      \cline{5-8}
       &
       &
       &
       &
      39 &
      7800 &
        4945.5 &
        0 &
      
         &
        
      \\
      \hline
      \multirow{2}{*}{A2} &
      \multirow{2}{*}{4} &
      \multirow{2}{*}{1680} &
      \multirow{2}{*}{280} &
      37 &
      7800 &
        7804.742 &
        0 &
      \multirow{2}{*}{9000} &
        \multirow{2}{*}{10597.588} &
        \multirow{2}{*}{0}
      \\
      \cline{5-8}
       &
       &
       &
       &
      39 &
      12000 &
        10597.588 &
        0 &
      
         &
        
      \\
      \hline
      \multirow{2}{*}{B1} &
      \multirow{2}{*}{7} &
      \multirow{2}{*}{120} &
      \multirow{2}{*}{16} &
      37 &
      12600 &
        10715.218 &
        0 &
      \multirow{2}{*}{13800} &
        \multirow{2}{*}{13830.149} &
        \multirow{2}{*}{0}
      \\
      \cline{5-8}
       &
       &
       &
       &
      0 &
      17400 &
        13830.149 &
        0 &
      
         &
        
      \\
      \hline
      \multirow{2}{*}{B2} &
      \multirow{2}{*}{0} &
      \multirow{2}{*}{120} &
      \multirow{2}{*}{16} &
      37 &
      15600 &
        14018.112 &
        0 &
      \multirow{2}{*}{16800} &
        \multirow{2}{*}{17001.87} &
        \multirow{2}{*}{0}
      \\
      \cline{5-8}
       &
       &
       &
       &
      39 &
      19800 &
        17001.87 &
        0 &
      
         &
        
      \\
      \hline
      \multirow{2}{*}{B3} &
      \multirow{2}{*}{0} &
      \multirow{2}{*}{1920} &
      \multirow{2}{*}{266} &
      37 &
      40200 &
        41387.661 &
        0 &
      \multirow{2}{*}{42000} &
        \multicolumn{2}{c}{\multirow{2}{*}{N/A}}
      \\
      \cline{5-8}
       &
       &
       &
       &
      39 &
      45000 &
        \multicolumn{2}{|c||}{N/A} &
      
        
      \\
      \hline
      \multirow{2}{*}{C1} &
      \multirow{2}{*}{0} &
      \multirow{2}{*}{0} &
      \multirow{2}{*}{0} &
      37 &
      19200 &
        17598 &
        0 &
      \multirow{2}{*}{21600} &
        \multirow{2}{*}{20970} &
        \multirow{2}{*}{0}
      \\
      \cline{5-8}
       &
       &
       &
       &
      39 &
      24600 &
        20970 &
        0 &
      
         &
        
      \\
      \hline
      \multirow{2}{*}{C2} &
      \multirow{2}{*}{0} &
      \multirow{2}{*}{120} &
      \multirow{2}{*}{13} &
      37 &
      27000 &
        25146.431 &
        0 &
      \multirow{2}{*}{28800} &
        \multirow{2}{*}{28754.147} &
        \multirow{2}{*}{0}
      \\
      \cline{5-8}
       &
       &
       &
       &
      39 &
      33000 &
        28754.147 &
        0 &
      
         &
        
      \\
      \hline
      \multirow{2}{*}{D1} &
      \multirow{2}{*}{0} &
      \multirow{2}{*}{0} &
      \multirow{2}{*}{0} &
      37 &
      29400 &
        26334.852 &
        0 &
      \multirow{2}{*}{31200} &
        \multirow{2}{*}{30903.416} &
        \multirow{2}{*}{0}
      \\
      \cline{5-8}
       &
       &
       &
       &
      0 &
      36600 &
        30903.416 &
        0 &
      
         &
        
      \\
      \hline
      \multirow{2}{*}{D2} &
      \multirow{2}{*}{15} &
      \multirow{2}{*}{8820} &
      \multirow{2}{*}{735} &
      37 &
      21600 &
        27583.649 &
        0 &
      \multirow{2}{*}{23400} &
        \multirow{2}{*}{32159.076} &
        \multirow{2}{*}{0}
      \\
      \cline{5-8}
       &
       &
       &
       &
      0 &
      28200 &
        32159.076 &
        0 &
      
         &
        
      \\
      \hline
      \multirow{2}{*}{D3} &
      \multirow{2}{*}{5} &
      \multirow{2}{*}{420} &
      \multirow{2}{*}{35} &
      37 &
      35400 &
        33288.738 &
        0 &
      \multirow{2}{*}{37200} &
        \multirow{2}{*}{37204.66} &
        \multirow{2}{*}{0}
      \\
      \cline{5-8}
       &
       &
       &
       &
      0 &
      41400 &
        37204.66 &
        0 &
      
         &
        
      \\
      \hline
      \multirow{2}{*}{E1} &
      \multirow{2}{*}{0} &
      \multirow{2}{*}{10500} &
      \multirow{2}{*}{437} &
      37 &
      36000 &
        \multicolumn{2}{|c||}{N/A} &
      \multirow{2}{*}{39000} &
        \multicolumn{2}{c}{\multirow{2}{*}{N/A}}
      \\
      \cline{5-8}
       &
       &
       &
       &
      39 &
      44400 &
        \multicolumn{2}{|c||}{N/A} &
      
        
      \\
      \hline
      \multirow{2}{*}{F1} &
      \multirow{2}{*}{0} &
      \multirow{2}{*}{0} &
      \multirow{2}{*}{0} &
      37 &
      57600 &
        \multicolumn{2}{|c||}{N/A} &
      \multirow{2}{*}{63000} &
        \multicolumn{2}{c}{\multirow{2}{*}{N/A}}
      \\
      \cline{5-8}
       &
       &
       &
       &
      0 &
      75000 &
        \multicolumn{2}{|c||}{N/A} &
      
        
      \\
\end{tabular}
\label{table:RASDATASET1} 
\end{sidewaystable}
\begin{sidewaystable}
\footnotesize
\caption{Resolved system ``RAS DATA SET 2'', costing \$8255. Seed: 1882575683.}
\centering
\begin{tabular}{c||c|c|c||c|c|c|c||c|c|c}
  \hline \hline
  &
  Unpref. & 
  Delay &
  Pty &
  Node &
  SA &
  Actual &
  Pty &
  TWT &
  Actual &
  Pty \\
      \hline
      \multirow{2}{*}{A1} &
      \multirow{2}{*}{0} &
      \multirow{2}{*}{780} &
      \multirow{2}{*}{130} &
      37 &
      3600 &
        3505.5 &
        0 &
      \multirow{2}{*}{5400} &
        \multirow{2}{*}{6499.5} &
        \multirow{2}{*}{0}
      \\
      \cline{5-8}
       &
       &
       &
       &
      39 &
      7800 &
        6499.5 &
        0 &
      
         &
        
      \\
      \hline
      \multirow{2}{*}{A2} &
      \multirow{2}{*}{0} &
      \multirow{2}{*}{60} &
      \multirow{2}{*}{10} &
      37 &
      11400 &
        10165.583 &
        0 &
      \multirow{2}{*}{12600} &
        \multirow{2}{*}{12842.851} &
        \multirow{2}{*}{0}
      \\
      \cline{5-8}
       &
       &
       &
       &
      39 &
      15600 &
        12842.851 &
        0 &
      
         &
        
      \\
      \hline
      \multirow{2}{*}{A3} &
      \multirow{2}{*}{18} &
      \multirow{2}{*}{3627} &
      \multirow{2}{*}{604} &
      37 &
      18000 &
        20482.5 &
        0 &
      \multirow{2}{*}{19800} &
        \multirow{2}{*}{23746.5} &
        \multirow{2}{*}{0}
      \\
      \cline{5-8}
       &
       &
       &
       &
      39 &
      22200 &
        23746.5 &
        0 &
      
         &
        
      \\
      \hline
      \multirow{2}{*}{A4} &
      \multirow{2}{*}{0} &
      \multirow{2}{*}{7440} &
      \multirow{2}{*}{1240} &
      37 &
      31800 &
        \multicolumn{2}{|c||}{N/A} &
      \multirow{2}{*}{39000} &
        \multicolumn{2}{c}{\multirow{2}{*}{N/A}}
      \\
      \cline{5-8}
       &
       &
       &
       &
      0 &
      36600 &
        \multicolumn{2}{|c||}{N/A} &
      
        
      \\
      \hline
      \multirow{2}{*}{B1} &
      \multirow{2}{*}{0} &
      \multirow{2}{*}{1560} &
      \multirow{2}{*}{216} &
      37 &
      4800 &
        7801.758 &
        0 &
      \multirow{2}{*}{9600} &
        \multirow{2}{*}{11331.87} &
        \multirow{2}{*}{0}
      \\
      \cline{5-8}
       &
       &
       &
       &
      39 &
      9000 &
        11331.87 &
        0 &
      
         &
        
      \\
      \hline
      \multirow{2}{*}{B2} &
      \multirow{2}{*}{7} &
      \multirow{2}{*}{60} &
      \multirow{2}{*}{8} &
      37 &
      26400 &
        31798.125 &
        0 &
      \multirow{2}{*}{35400} &
        \multirow{2}{*}{35458.125} &
        \multirow{2}{*}{0}
      \\
      \cline{5-8}
       &
       &
       &
       &
      39 &
      31200 &
        35458.125 &
        0 &
      
         &
        
      \\
      \hline
      \multirow{2}{*}{B3} &
      \multirow{2}{*}{12} &
      \multirow{2}{*}{9900} &
      \multirow{2}{*}{1375} &
      37 &
      11400 &
        17107.716 &
        0 &
      \multirow{2}{*}{10800} &
        \multirow{2}{*}{20534.147} &
        \multirow{2}{*}{0}
      \\
      \cline{5-8}
       &
       &
       &
       &
      0 &
      16800 &
        20534.147 &
        0 &
      
         &
        
      \\
      \hline
      \multirow{2}{*}{C1} &
      \multirow{2}{*}{5} &
      \multirow{2}{*}{120} &
      \multirow{2}{*}{13} &
      37 &
      26400 &
        35475.007 &
        104 &
      \multirow{2}{*}{39000} &
        \multirow{2}{*}{38949.474} &
        \multirow{2}{*}{0}
      \\
      \cline{5-8}
       &
       &
       &
       &
      39 &
      31800 &
        38949.474 &
        0 &
      
         &
        
      \\
      \hline
      \multirow{2}{*}{C2} &
      \multirow{2}{*}{0} &
      \multirow{2}{*}{0} &
      \multirow{2}{*}{0} &
      37 &
      0 &
        198 &
        0 &
      \multirow{2}{*}{3600} &
        \multirow{2}{*}{3570} &
        \multirow{2}{*}{0}
      \\
      \cline{5-8}
       &
       &
       &
       &
      39 &
      5400 &
        3570 &
        0 &
      
         &
        
      \\
      \hline
      \multirow{2}{*}{C3} &
      \multirow{2}{*}{8} &
      \multirow{2}{*}{60} &
      \multirow{2}{*}{6} &
      37 &
      20400 &
        21773.144 &
        0 &
      \multirow{2}{*}{25200} &
        \multirow{2}{*}{25199.575} &
        \multirow{2}{*}{0}
      \\
      \cline{5-8}
       &
       &
       &
       &
      0 &
      25800 &
        25199.575 &
        0 &
      
         &
        
      \\
      \hline
      \multirow{2}{*}{D1} &
      \multirow{2}{*}{6} &
      \multirow{2}{*}{180} &
      \multirow{2}{*}{15} &
      37 &
      43800 &
        40795.5 &
        0 &
      \multirow{2}{*}{44400} &
        \multicolumn{2}{c}{\multirow{2}{*}{N/A}}
      \\
      \cline{5-8}
       &
       &
       &
       &
      39 &
      51000 &
        \multicolumn{2}{|c||}{N/A} &
      
        
      \\
      \hline
      \multirow{2}{*}{D2} &
      \multirow{2}{*}{0} &
      \multirow{2}{*}{0} &
      \multirow{2}{*}{0} &
      37 &
      3600 &
        1855.382 &
        0 &
      \multirow{2}{*}{6600} &
        \multirow{2}{*}{5735.068} &
        \multirow{2}{*}{0}
      \\
      \cline{5-8}
       &
       &
       &
       &
      39 &
      9600 &
        5735.068 &
        0 &
      
         &
        
      \\
      \hline
      \multirow{2}{*}{E1} &
      \multirow{2}{*}{23} &
      \multirow{2}{*}{900} &
      \multirow{2}{*}{37} &
      37 &
      6600 &
        5492.184 &
        0 &
      \multirow{2}{*}{9600} &
        \multirow{2}{*}{10075.095} &
        \multirow{2}{*}{0}
      \\
      \cline{5-8}
       &
       &
       &
       &
      39 &
      14400 &
        10075.095 &
        0 &
      
         &
        
      \\
      \hline
      \multirow{2}{*}{E2} &
      \multirow{2}{*}{0} &
      \multirow{2}{*}{8040} &
      \multirow{2}{*}{334} &
      37 &
      1800 &
        9174.858 &
        0 &
      \multirow{2}{*}{7200} &
        \multirow{2}{*}{15266.294} &
        \multirow{2}{*}{0}
      \\
      \cline{5-8}
       &
       &
       &
       &
      0 &
      10800 &
        15266.294 &
        0 &
      
         &
        
      \\
      \hline
      \multirow{2}{*}{E3} &
      \multirow{2}{*}{0} &
      \multirow{2}{*}{40620} &
      \multirow{2}{*}{1692} &
      37 &
      9000 &
        \multicolumn{2}{|c||}{N/A} &
      \multirow{2}{*}{12000} &
        \multicolumn{2}{c}{\multirow{2}{*}{N/A}}
      \\
      \cline{5-8}
       &
       &
       &
       &
      0 &
      18000 &
        \multicolumn{2}{|c||}{N/A} &
      
        
      \\
      \hline
      \multirow{2}{*}{E4} &
      \multirow{2}{*}{0} &
      \multirow{2}{*}{21600} &
      \multirow{2}{*}{900} &
      37 &
      30000 &
        \multicolumn{2}{|c||}{N/A} &
      \multirow{2}{*}{32400} &
        \multicolumn{2}{c}{\multirow{2}{*}{N/A}}
      \\
      \cline{5-8}
       &
       &
       &
       &
      0 &
      37800 &
        \multicolumn{2}{|c||}{N/A} &
      
        
      \\
      \hline
      \multirow{2}{*}{F1} &
      \multirow{2}{*}{0} &
      \multirow{2}{*}{19140} &
      \multirow{2}{*}{531} &
      37 &
      29400 &
        \multicolumn{2}{|c||}{N/A} &
      \multirow{2}{*}{33600} &
        \multicolumn{2}{c}{\multirow{2}{*}{N/A}}
      \\
      \cline{5-8}
       &
       &
       &
       &
      39 &
      41400 &
        \multicolumn{2}{|c||}{N/A} &
      
        
      \\
      \hline
      \multirow{2}{*}{F2} &
      \multirow{2}{*}{0} &
      \multirow{2}{*}{34620} &
      \multirow{2}{*}{961} &
      37 &
      30000 &
        \multicolumn{2}{|c||}{N/A} &
      \multirow{2}{*}{36000} &
        \multicolumn{2}{c}{\multirow{2}{*}{N/A}}
      \\
      \cline{5-8}
       &
       &
       &
       &
      0 &
      51000 &
        \multicolumn{2}{|c||}{N/A} &
      
        
      \\
\end{tabular}
\label{table:RASDATASET2} 
\end{sidewaystable}
\begin{sidewaystable}
\footnotesize
\caption{Resolved system ``RAS DATA SET 3'', costing \$10777. Seed: -2075839653.}
\centering
\begin{tabular}{c||c|c|c||c|c|c|c||c|c|c}
  \hline \hline
  &
  Unpref. & 
  Delay &
  Pty &
  Node &
  SA &
  Actual &
  Pty &
  TWT &
  Actual &
  Pty \\
      \hline
      \multirow{2}{*}{A1} &
      \multirow{2}{*}{0} &
      \multirow{2}{*}{0} &
      \multirow{2}{*}{0} &
      37 &
      2400 &
        967.5 &
        0 &
      \multirow{2}{*}{4200} &
        \multirow{2}{*}{3514.5} &
        \multirow{2}{*}{0}
      \\
      \cline{5-8}
       &
       &
       &
       &
      39 &
      6000 &
        3514.5 &
        0 &
      
         &
        
      \\
      \hline
      \multirow{2}{*}{A2} &
      \multirow{2}{*}{7} &
      \multirow{2}{*}{5640} &
      \multirow{2}{*}{940} &
      37 &
      2400 &
        6649.715 &
        0 &
      \multirow{2}{*}{4200} &
        \multirow{2}{*}{9821.148} &
        \multirow{2}{*}{0}
      \\
      \cline{5-8}
       &
       &
       &
       &
      0 &
      6600 &
        9821.148 &
        0 &
      
         &
        
      \\
      \hline
      \multirow{2}{*}{A3} &
      \multirow{2}{*}{10} &
      \multirow{2}{*}{0} &
      \multirow{2}{*}{0} &
      37 &
      22800 &
        20900.576 &
        0 &
      \multirow{2}{*}{24000} &
        \multirow{2}{*}{23641.724} &
        \multirow{2}{*}{0}
      \\
      \cline{5-8}
       &
       &
       &
       &
      0 &
      27000 &
        23641.724 &
        0 &
      
         &
        
      \\
      \hline
      \multirow{2}{*}{A4} &
      \multirow{2}{*}{21} &
      \multirow{2}{*}{600} &
      \multirow{2}{*}{100} &
      37 &
      33600 &
        36100.967 &
        0 &
      \multirow{2}{*}{39000} &
        \multirow{2}{*}{39319.634} &
        \multirow{2}{*}{0}
      \\
      \cline{5-8}
       &
       &
       &
       &
      0 &
      38400 &
        39319.634 &
        0 &
      
         &
        
      \\
      \hline
      \multirow{2}{*}{A5} &
      \multirow{2}{*}{0} &
      \multirow{2}{*}{12000} &
      \multirow{2}{*}{1999} &
      37 &
      32400 &
        \multicolumn{2}{|c||}{N/A} &
      \multirow{2}{*}{34200} &
        \multicolumn{2}{c}{\multirow{2}{*}{N/A}}
      \\
      \cline{5-8}
       &
       &
       &
       &
      39 &
      36600 &
        \multicolumn{2}{|c||}{N/A} &
      
        
      \\
      \hline
      \multirow{1}{*}{B1} &
      \multirow{1}{*}{0} &
      \multirow{1}{*}{3720} &
      \multirow{1}{*}{516} &
      0 &
      1800 &
        5206.284 &
        0 &
      \multirow{1}{*}{1200} &
        \multirow{1}{*}{5206.284} &
        \multirow{1}{*}{0}
      \\
      \hline
      \multirow{1}{*}{B2} &
      \multirow{1}{*}{0} &
      \multirow{1}{*}{0} &
      \multirow{1}{*}{0} &
      39 &
      4800 &
        2869.163 &
        0 &
      \multirow{1}{*}{3000} &
        \multirow{1}{*}{2869.163} &
        \multirow{1}{*}{0}
      \\
      \hline
      \multirow{2}{*}{B3} &
      \multirow{2}{*}{9} &
      \multirow{2}{*}{240} &
      \multirow{2}{*}{33} &
      37 &
      -3000 &
        2694.95 &
        0 &
      \multirow{2}{*}{6000} &
        \multirow{2}{*}{5871.176} &
        \multirow{2}{*}{0}
      \\
      \cline{5-8}
       &
       &
       &
       &
      39 &
      1800 &
        5871.176 &
        0 &
      
         &
        
      \\
      \hline
      \multirow{2}{*}{B4} &
      \multirow{2}{*}{4} &
      \multirow{2}{*}{6485.141} &
      \multirow{2}{*}{900} &
      37 &
      16800 &
        28769.145 &
        264 &
      \multirow{2}{*}{32400} &
        \multirow{2}{*}{38680.717} &
        \multirow{2}{*}{0}
      \\
      \cline{5-8}
       &
       &
       &
       &
      0 &
      21600 &
        38680.717 &
        548 &
      
         &
        
      \\
      \hline
      \multirow{1}{*}{C1} &
      \multirow{1}{*}{0} &
      \multirow{1}{*}{5580} &
      \multirow{1}{*}{620} &
      0 &
      6000 &
        8922.853 &
        0 &
      \multirow{1}{*}{4200} &
        \multirow{1}{*}{8922.853} &
        \multirow{1}{*}{0}
      \\
      \hline
      \multirow{2}{*}{C2} &
      \multirow{2}{*}{11} &
      \multirow{2}{*}{540} &
      \multirow{2}{*}{60} &
      37 &
      0 &
        3966.431 &
        0 &
      \multirow{2}{*}{7200} &
        \multirow{2}{*}{7574.147} &
        \multirow{2}{*}{0}
      \\
      \cline{5-8}
       &
       &
       &
       &
      39 &
      6000 &
        7574.147 &
        0 &
      
         &
        
      \\
      \hline
      \multirow{2}{*}{C3} &
      \multirow{2}{*}{13} &
      \multirow{2}{*}{10093.222} &
      \multirow{2}{*}{1121} &
      37 &
      30000 &
        \multicolumn{2}{|c||}{N/A} &
      \multirow{2}{*}{37200} &
        \multicolumn{2}{c}{\multirow{2}{*}{N/A}}
      \\
      \cline{5-8}
       &
       &
       &
       &
      0 &
      36000 &
        \multicolumn{2}{|c||}{N/A} &
      
        
      \\
      \hline
      \multirow{2}{*}{D1} &
      \multirow{2}{*}{0} &
      \multirow{2}{*}{900} &
      \multirow{2}{*}{75} &
      37 &
      4200 &
        4913.996 &
        0 &
      \multirow{2}{*}{7800} &
        \multirow{2}{*}{8793.682} &
        \multirow{2}{*}{0}
      \\
      \cline{5-8}
       &
       &
       &
       &
      39 &
      10200 &
        8793.682 &
        0 &
      
         &
        
      \\
      \hline
      \multirow{2}{*}{D2} &
      \multirow{2}{*}{0} &
      \multirow{2}{*}{5340} &
      \multirow{2}{*}{444} &
      37 &
      42000 &
        \multicolumn{2}{|c||}{N/A} &
      \multirow{2}{*}{43800} &
        \multicolumn{2}{c}{\multirow{2}{*}{N/A}}
      \\
      \cline{5-8}
       &
       &
       &
       &
      39 &
      48000 &
        \multicolumn{2}{|c||}{N/A} &
      
        
      \\
      \hline
      \multirow{2}{*}{E1} &
      \multirow{2}{*}{0} &
      \multirow{2}{*}{39780} &
      \multirow{2}{*}{1657} &
      37 &
      10200 &
        \multicolumn{2}{|c||}{N/A} &
      \multirow{2}{*}{13200} &
        \multicolumn{2}{c}{\multirow{2}{*}{N/A}}
      \\
      \cline{5-8}
       &
       &
       &
       &
      0 &
      19200 &
        \multicolumn{2}{|c||}{N/A} &
      
        
      \\
      \hline
      \multirow{2}{*}{E2} &
      \multirow{2}{*}{0} &
      \multirow{2}{*}{1140} &
      \multirow{2}{*}{47} &
      37 &
      18000 &
        17157 &
        0 &
      \multirow{2}{*}{21000} &
        \multirow{2}{*}{22299} &
        \multirow{2}{*}{0}
      \\
      \cline{5-8}
       &
       &
       &
       &
      39 &
      26400 &
        22299 &
        0 &
      
         &
        
      \\
      \hline
      \multirow{2}{*}{E3} &
      \multirow{2}{*}{7} &
      \multirow{2}{*}{360} &
      \multirow{2}{*}{15} &
      37 &
      21000 &
        19022.184 &
        0 &
      \multirow{2}{*}{24000} &
        \multirow{2}{*}{24280.366} &
        \multirow{2}{*}{0}
      \\
      \cline{5-8}
       &
       &
       &
       &
      39 &
      28800 &
        24280.366 &
        0 &
      
         &
        
      \\
      \hline
      \multirow{2}{*}{E4} &
      \multirow{2}{*}{0} &
      \multirow{2}{*}{10800} &
      \multirow{2}{*}{450} &
      37 &
      40800 &
        \multicolumn{2}{|c||}{N/A} &
      \multirow{2}{*}{43800} &
        \multicolumn{2}{c}{\multirow{2}{*}{N/A}}
      \\
      \cline{5-8}
       &
       &
       &
       &
      39 &
      49800 &
        \multicolumn{2}{|c||}{N/A} &
      
        
      \\
      \hline
      \multirow{2}{*}{F1} &
      \multirow{2}{*}{15} &
      \multirow{2}{*}{13205.145} &
      \multirow{2}{*}{366} &
      37 &
      18000 &
        24468 &
        0 &
      \multirow{2}{*}{23400} &
        \multirow{2}{*}{35432.569} &
        \multirow{2}{*}{25}
      \\
      \cline{5-8}
       &
       &
       &
       &
      0 &
      34800 &
        35432.569 &
        0 &
      
         &
        
      \\
      \hline
      \multirow{2}{*}{F2} &
      \multirow{2}{*}{0} &
      \multirow{2}{*}{18000} &
      \multirow{2}{*}{500} &
      37 &
      37200 &
        \multicolumn{2}{|c||}{N/A} &
      \multirow{2}{*}{42000} &
        \multicolumn{2}{c}{\multirow{2}{*}{N/A}}
      \\
      \cline{5-8}
       &
       &
       &
       &
      39 &
      51000 &
        \multicolumn{2}{|c||}{N/A} &
      
        
      \\
\end{tabular}
\label{table:RASDATASET3} 
\end{sidewaystable}
\begin{sidewaystable}
\footnotesize
\caption{Resolved system ``RAS DATA SET TOY'', costing \$821. Seed: -532028618.}
\centering
\begin{tabular}{c||c|c|c||c|c|c|c||c|c|c}
  \hline \hline
  &
  Unpref. & 
  Delay &
  Pty &
  Node &
  SA &
  Actual &
  Pty &
  TWT &
  Actual &
  Pty \\
      \hline
      \multirow{2}{*}{A1} &
      \multirow{2}{*}{0} &
      \multirow{2}{*}{1200} &
      \multirow{2}{*}{199} &
      6 &
      2400 &
        4200 &
        0 &
      \multirow{2}{*}{8700} &
        \multirow{2}{*}{6000} &
        \multirow{2}{*}{0}
      \\
      \cline{5-8}
       &
       &
       &
       &
      12 &
      4800 &
        6000 &
        0 &
      
         &
        
      \\
      \hline
      \multirow{2}{*}{B1} &
      \multirow{2}{*}{0} &
      \multirow{2}{*}{3600} &
      \multirow{2}{*}{500} &
      6 &
      -3000 &
        6403.16 &
        122 &
      \multirow{2}{*}{4800} &
        \multirow{2}{*}{8823.328} &
        \multirow{2}{*}{0}
      \\
      \cline{5-8}
       &
       &
       &
       &
      0 &
      5400 &
        8823.328 &
        0 &
      
         &
        
      \\
      \hline
      \multirow{2}{*}{C1} &
      \multirow{2}{*}{0} &
      \multirow{2}{*}{0} &
      \multirow{2}{*}{0} &
      6 &
      3000 &
        2400 &
        0 &
      \multirow{2}{*}{7200} &
        \multirow{2}{*}{4800} &
        \multirow{2}{*}{0}
      \\
      \cline{5-8}
       &
       &
       &
       &
      12 &
      6000 &
        4800 &
        0 &
      
         &
        
      \\
\end{tabular}
\label{table:RASDATASETTOY} 
\end{sidewaystable}

\end{document}
